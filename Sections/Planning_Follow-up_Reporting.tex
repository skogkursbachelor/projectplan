\section{Planning, Follow-up, \& Reporting}
\subsection{Selection of SDLC model}

\begin{comment}
    Describe how the group will follow the chosen model
    
    Fire viktige forhold å ta hensyn til ved valg av utviklingmodell /
    prosessrammeverk :
    ▪ Universitetets krav til prosjektarbeidet/bacheloroppgaven
    ▪ Karakteristika ved oppgaven
    ▪ Motivasjonen og ferdighetene til deltagerne
    ▪ Ønsker og krav fra oppdragsgiver
    Velg modell – deretter bør dere tilpasse den / sette den opp til deres
    “setting”.

    % etter et par sprints ønsker vi en MVP, Hva er MVP?

    % Nevne andre metoder og hvorfor vi ikke valgte dem
\end{comment}

For this project, the team needed to select an appropriate system development life cycle (SDLC) model. Key factors considered in this decision included the clarity and flexibility of the requirements, team size, project timeline, product delivery goals, and prior experience \cite{sdlc_model}. 

The requirements provided by the Product Owner are intentionally ambiguous and flexible, allowing the team to prioritize the few fixed requirements upfront while iteratively refining the flexible ones over time. The team is composed of two members with similar experience levels, enabling close collaboration and effective decision-making. With a short project timeline of four months, a structure of 2-week sprints aligns perfectly with Scrum's iterative and adaptive approach, ensuring steady progress and regular opportunities for feedback. 

%After careful consideration of these factors, we determined that Scrum was the best fit for the team and the project. 
%Skrive om hvorfor ikke waterfall etc
Since our project does not have clearly defined requirements and collaboration with the Product Owner was needed, Scrum was the best fit for the team and the project \cite{sdlc_model}. To implement Scrum effectively, several key practices will be integrated into our workflow \cite{scrum_guide}.

\begin{itemize}
    \item \textbf{Sprint Meetings:} The project will be divided into 2-week sprints. At the start of each sprint, sprint meetings will include sprint planning, reviews, and retrospectives. During the planning phase, the team will select tasks from the product backlog to form the sprint backlog for the upcoming sprint. The review phase will focus on assessing progress and determining whether adjustments to the product backlog are needed. The retrospective phase will identify areas for improvement in the Scrum process itself.
    \item \textbf{Daily Scrum Meetings:} Short daily meetings will be conducted to discuss the progress of ongoing tasks, identify potential obstacles, and ensure alignment between team members.
    \item \textbf{Kanban:} To complement Scrum and further enhance workflow visibility, the team will utilize a Kanban board to track and manage tasks in GitHub. The Kanban board will consist of columns representing different stages of the workflow, such as Product Backlog, Sprint Backlog, In Progress, In Review, Done, and Discarded.
    \item \textbf{Scrum Master:} Given that the team consists of only two members, we have decided to share the role of Scrum Master. Both members are responsible for ensuring adherence to the Scrum framework and continuously working to improve team efficiency. This collaborative approach allows us to maintain flexibility while upholding Scrum principles throughout the project.
\end{itemize}


% This workflow ensures a structured yet flexible approach to development, allowing the group to adapt to evolving requirements while maintaining steady progress toward project goals.
% Scrum Master ?

\subsection{Plan for Status Meetings}
Overview of the regular two-week status meeting cycle. For a more detailed view of all sprint meetings, see \hyperref[fig:gantt]{Figure \ref*{fig:gantt}}.

\begin{table}[H]
    \centering
    \begin{tabularx}{\textwidth}{|c|c|>{\centering\arraybackslash}X|c|c|c|}
    \hline
    \textbf{Week} & \textbf{Monday} & \textbf{Tuesday} & \textbf{Wednesday} & \textbf{Thursday} & \textbf{Friday} \\
    \hline
    1 & Sprint Meeting & \makecell{Daily Scrum \\ Supervisor (12:00)} & Daily Scrum & Daily 
    Scrum & Daily Scrum \\
    \hline
    2 & Daily Scrum & \makecell{Daily Scrum \\ Supervisor (12:00)} & Daily Scrum & Daily 
    Scrum & Daily Scrum \\
    \hline
    \end{tabularx}
    \caption{Plan for Status Meetings}
    \label{tab:meeting_plan}
\end{table}

\newpage
