\section{Quality Assurance Organization}

\subsection{Documentation, Storage, Source Code} \label{sec:documentation_storage_source_code}

\begin{comment}
    Documentation
        Andre skal kanskje bruke eller gjenbruke produktet vårt, så viktig med dokumentasjon av både sluttproduktet, i tillegg til hele prosessen.

        Kode-produkter(?)
            Deployment
            APIdoc

        Arbeidsprosessen
            Så mye som mulig skal ligge under issues under GitHub Project, slik at vi dokumenterer hva vi har gjort. Timeføring blir gjort i Traggo som er tag-basert, slik at vi kan se hvor lenge vi har jobbet med feks admin, thesis, metting osv.
        
        Møter og enigheter
            Møtenotater/referat skal bli tatt av alle møter
            Alle viktige valg som blir gjort etter diskusjoner blir notert
        
    Storage ?
        Skylagring i de SaaS vi bruker (Overleaf, Teamgantt, Traggo(?), fler) med backup i git for latex dokumenter og annet fil-basert data, google drive for dokumenter. I tilegg til lokal lagring på hvert medlems pc

    Source code
        Dokumentasjon av koden kommer til å bli brukt av gruppemedlemene, men kan også bli brukt av noen som vil videreutvikle produktet.
        Følge industri-standard dokumentasjon og kodestil for hvert kodespråk, med hyppig bruk av en linter. 
        Prøve å skrive self-documenting code ved å ha descriptive funksjonsnavn og variabler, slik at man kan holde kodekommentarene korte og konsise
        
\end{comment}

\subsection{Plan for Inspections \& Testing}\label{sec:plan_for_inspections_&_testing}

\begin{comment}
    - GitOps som et sentralt konsept
    - Bruke CICD Pipelines for å automatisere så mye som mulig
    f.eks. statiske kodesjekker og inspections, unittester og integrasjonstester, også deployment for å kunne ha manuelle inspeksjoner og tester during development.
    - Linting for Go (eks. golangci-lint).
    - Plan om usertesting etter vi har MVP på plass midt i mars, jobbe utifra kravspesifikasjon og tilbakemelding fra productowner fram til det.
    
\end{comment}

\subsection{Tools}\label{sec:tools}
Overview of the tools that the group plans to use during the project.
\begin{table} [H]
    \centering
    \begin{tabular}{|l|l|l|}
    \hline
    Name & Type & Use-case \\
    \hline
    OverLeaf & Online \LaTeX-Editor & Report writing \\
    JetBrains IDE Family & IDE & Code Development \\
    VS Code & IDE & Code Development \\
    GitHub & Code Repository & Version Control \& Scrum \\
    Docker & Containerization Platform & Containerizing deployment \\
    OpenStack SkyHigh & Cloud Computing Platform & Deployment \\
    Postman & API Collaboration Platform & API Development \\
    TeamGantt & Gantt Diagram Tool & Gantt Chart \\
    draw.io & Diagramming Tool & Other Diagrams \\
    Traggo & Time-Tracking Tool & Tracking Time \\
    Slack & Communication Platform & Communication \\
    Leaflet / OpenLayers & JavaScript Library for Interactive Maps & Interactive Map  \\
     & Database & \\
    \hline
    \end{tabular}
    \caption{Tools}
    \label{tab:tools}
\end{table}

\subsection{Risk Analysis}\label{sec:risk_analysis}
\begin{table}[H]
    \centering
    \begin{tabular}{|c|c|c|c|c|c|}
    \hline
    Likelihood / Severity & Minimal & Marginal & Moderate & Major & Critical \\
    \hline
    Certain & \cellcolor[HTML]{ffC300} & \cellcolor[HTML]{ffC300} & \cellcolor[HTML]{ff4233} & \cellcolor[HTML]{ff4233} & \cellcolor[HTML]{ff4233} \\ 
    \hline
    Likely & \cellcolor[HTML]{fff000} & \cellcolor[HTML]{ffC300} & \cellcolor[HTML]{ffC300} & \cellcolor[HTML]{ff4233} & \cellcolor[HTML]{ff4233} \\ 
    \hline     
    Possible & \cellcolor[HTML]{74ff00} & \cellcolor[HTML]{fff000} & \cellcolor[HTML]{ffC300} & \cellcolor[HTML]{ffC300} & \cellcolor[HTML]{ff4233} \\
    \hline
    Unlikely & \cellcolor[HTML]{74ff00} & \cellcolor[HTML]{74ff00} & \cellcolor[HTML]{fff000} & \cellcolor[HTML]{ffC300} & \cellcolor[HTML]{ffC300} \\ 
    \hline
    Rare & \cellcolor[HTML]{74ff00} & \cellcolor[HTML]{74ff00} & \cellcolor[HTML]{fff000} & \cellcolor[HTML]{fff000} & \cellcolor[HTML]{ffC300} \\
    \hline
    \end{tabular}
    \caption{Risk Matrix (5x5)}
    \label{tab:risk_matrix}
\end{table}

\begin{comment}
    Mulige Risks

    Gruppemedlem blir syk, se tabell
    Miste source code eller dokumenter, som gjør at vi må recovere fra backups
    Infrastruktur og systemer blir utsatt for cyberangrep, typ ddos
    
\end{comment}

\begin{table}[H]
    \centering
    \begin{tabular}{|c|c|c|c|}
    \hline
    Risk \# & Description & Likelihood & Consequence \\
    \hline
    1 & \makecell{One or more group members get sick over a longer \\period, making it hard to work on the project.} & Unlikely \cellcolor[HTML]{ffC300} & Major \cellcolor[HTML]{ffC300} \\
    \hline
    2 & \makecell{Source code and/or documents become lost and unrecoverable \\due to  issues with their storage location.} & Unlikely \cellcolor[HTML]{ffC300} & Major \cellcolor[HTML]{ffC300} \\
    \hline
    3 & \makecell{Infrastructure and systems could be attacked in a cyber attack, \\compromising the availability of the systems. } & Unlikely \cellcolor[HTML]{ffC300} & Major\cellcolor[HTML]{ffC300} \\
    \hline
    4 & & \cellcolor[HTML]{74ff00} & \cellcolor[HTML]{74ff00} \\
    \hline
    \end{tabular}
    \caption{Risks}
    \label{tab:risks}
\end{table}

\begin{table}[H]
    \centering
    \begin{tabular}{|c|c|c|}
    \hline
    Risk \# & Mitigation \& Measures & Priority \\
    \hline
    1 & \makecell{The sick group member should work as much as possible. \\ Good documentation and communication will make it easier for \\ the other member to step in and help with the workload.} & Medium \cellcolor[HTML]{fff000} \\
    \hline
    2 & \makecell{Backups of all work should be made and stored in at least one \\separate location. See \ref{sec:documentation_storage_source_code}.} & High \cellcolor[HTML]{ff4233} \\
    \hline
    3 & & Low \cellcolor[HTML]{fff000} \\
    \hline
    4 & & xxx \cellcolor[HTML]{74ff00} \\
    \hline
    \end{tabular}
    \caption{Risk Mitigation \& Measures}
    \label{tab:risk_mitigation}
\end{table}