\section{Quality Assurance Organization}

\subsection{Documentation, Source Code, Storage} \label{sec:documentation_storage_source_code}

\begin{comment}
    Documentation
        Andre skal kanskje bruke eller gjenbruke produktet vårt, så viktig med dokumentasjon av både sluttproduktet, i tillegg til hele prosessen.

        Commitmeldingsstandard

        Kode-produkter(?)
            Deployment
            APIdoc

        Arbeidsprosessen
            Så mye som mulig skal ligge under issues under GitHub Project, slik at vi dokumenterer hva vi har gjort. Timeføring blir gjort i Traggo som er tag-basert, slik at vi kan se hvor lenge vi har jobbet med feks admin, thesis, metting osv.
        
        Møter og enigheter
            Møtenotater/referat skal bli tatt av alle møter
            Alle viktige valg som blir gjort etter diskusjoner blir notert

    Source code
        Dokumentasjon av koden kommer til å bli brukt av gruppemedlemene, men kan også bli brukt av noen som vil videreutvikle produktet.
        Følge industri-standard dokumentasjon og kodestil for hvert kodespråk, med hyppig bruk av en linter. 
        Prøve å skrive self-documenting code ved å ha descriptive funksjonsnavn og variabler, slik at man kan holde kodekommentarene korte og konsise
        
\end{comment}
% Kanskje ha 5.1.1 til 5.1 og 5.1.2 til 5.2 osv
% \subsection{Documentation}
\subsubsection{Documentation}  
Proper documentation is crucial, as there is a possibility that others may use the code in the future. To address this, both the code and the development process will be thoroughly documented throughout the project. All tasks will be tracked and recorded as GitHub issues, ensuring a clear history of the project's progress. Additionally, the proceedings of all meetings will be recorded by the designated minutes-taker. Furthermore, a comprehensive user guide detailing how to use the product will be created.  

\subsubsection{Source Code}  
All source code will adhere to industry-standard documentation practices and linting rules specific to each programming language to ensure clarity and maintainability. For Go, the project will follow Google's official style guide \cite{google_go_style}. \\

GitHub commits should adhere to a defined standard \cite{commit_standard}. Each commit message must include a relevant keyword that categorizes the task performed, such as feat, fix, docs, style, refactor, test, or chore. Additionally, each message should provide a brief and concise description of the changes made.

\subsubsection{Storage}
% Skylagring i de SaaS vi bruker (Overleaf, Teamgantt, Traggo(?), fler) med backup i git for latex dokumenter og annet fil-basert data, google drive for dokumenter. I tilegg til lokal lagring på hvert medlems pc
% 3-2-1 rule med ref til noe.
% "offsite" blir da egene pcer, siden de har lang spatial distanse

% Liten time-scale så trenger ikke flere medium av samme plass
\begin{table} [H]
    \centering
    \begin{tabular}{|l|l|l|l|}
    \hline
    Asset & Main & Backup & Offsite \\
    \hline
    % noe annet for local pc
    % latex documenter blir også lagret på skyhigh siden vi autocloner alle repoer
    \LaTeX-documents & Overleaf & GitHub & Local PC \\
    % Kanban blir også tatt med i sourcecode
    Source code & GitHub & SkyHiGh Backup Server & Local PC \\
    Kanban with issues & GitHub & IKKE SKYHIGH & Local PC \\
    % Jobhours overview? noesånt
    Minutes & GitHub & SkyHiGh Backup Server & Local PC \\
    Traggo & SkyHiGh & SkyHiGh Backup Server & Local PC \\
    Slack data & & & \\
    \hline
    \end{tabular}
    \caption{3-2-1 Backup Strategy}
    \label{tab:tools}
\end{table}

\begin{comment}
    https://stackoverflow.com/questions/19576742/how-to-clone-all-repos-at-once-from-github
    for auto clone til skyhigh
\end{comment}

\subsection{Plan for Inspections \& Testing}\label{sec:plan_for_inspections_&_testing}

\begin{itemize}
    \item Implement CI/CD pipelines in GitHub to automate processes as much as possible, starting early in development.
    \item The pipelines will include:
    \begin{itemize}
        \item Static Code Analysis and Linting
        \item Unit and Integration Testing
        \item Deployment
        % Docker?
        % Packaging?
        % Security?
    \end{itemize}
    \item Conduct user testing with the Product Owner and potential end-users once the MVP is complete.
    \item For unit and integration testing, we aim for 80\% or higher coverage, with a focus on critical components and an emphasis on test quality over quantity.
\end{itemize}


\begin{comment}
    - GitOps som et sentralt konsept
    - Bruke CI/CD Pipelines i GitHubfor å automatisere så mye som mulig
    f.eks. statiske kodesjekker og inspections, unittester og integrasjonstester, også deployment for å kunne ha manuelle inspeksjoner og tester during development.
    - Linting for Go (eks. golangci-lint).
    - Plan om usertesting etter vi har MVP på plass midt i mars, jobbe utifra kravspesifikasjon og tilbakemelding fra productowner fram til det.
    
\end{comment}

\subsection{Tools}\label{sec:tools}
This section provides a detailed overview of the tools that the group intends to use throughout the project. These tools have been selected to facilitate various aspects of the project, including development, collaboration, deployment, and documentation. The table below outlines each tool, its type, and its specific use case within the project workflow.
\begin{table} [H]
    \centering
    \begin{tabular}{|l|l|l|}
    \hline
    Name & Type & Use-case \\
    \hline
    OverLeaf & Online \LaTeX-Editor & Report writing \\
    JetBrains IDE Family & IDE & Code Development \\
    VS Code & IDE & Code Development \\
    GitHub & Code Repository & Version Control \& Kanban \\
    Docker & Containerization Platform & Containerizing Deployment \\
    OpenStack SkyHigh & Cloud Computing Platform & Deployment \\
    Postman & API Collaboration Platform & API Development \\
    TeamGantt & Gantt Diagram Tool & Gantt Chart \\
    draw.io & Diagramming Tool & Other Diagrams \\
    Traggo & Time-Tracking Tool & Tracking Time \\
    Slack & Communication Platform & Communication \\
    Leaflet / OpenLayers & JavaScript Library for Interactive Maps & Interactive Map  \\
     & Database & \\
    \hline
    \end{tabular}
    \caption{Tools}
    \label{tab:tools}
\end{table}

\subsection{Risk Analysis}\label{sec:risk_analysis}

This section presents potential risk scenarios identified for the project. Each scenario is evaluated using a matrix that assesses likelihood and severity, categorized into four levels ranging from green (low risk) to red (high risk). Additionally, each scenario includes proposed measures or mitigations, prioritized from low to high.

\begin{table}[H]
    \centering
    \begin{tabular}{|c|c|c|c|c|c|}
    \hline
    Likelihood / Severity & Minimal & Marginal & Moderate & Major & Critical \\
    \hline
    Certain & \cellcolor[HTML]{ffC300} & \cellcolor[HTML]{ffC300} & \cellcolor[HTML]{ff4233} & \cellcolor[HTML]{ff4233} & \cellcolor[HTML]{ff4233} \\ 
    \hline
    Likely & \cellcolor[HTML]{fff000} & \cellcolor[HTML]{ffC300} & \cellcolor[HTML]{ffC300} & \cellcolor[HTML]{ff4233} & \cellcolor[HTML]{ff4233} \\ 
    \hline     
    Possible & \cellcolor[HTML]{74ff00} & \cellcolor[HTML]{fff000} & \cellcolor[HTML]{ffC300} & \cellcolor[HTML]{ffC300} & \cellcolor[HTML]{ff4233} \\
    \hline
    Unlikely & \cellcolor[HTML]{74ff00} & \cellcolor[HTML]{74ff00} & \cellcolor[HTML]{fff000} & \cellcolor[HTML]{ffC300} & \cellcolor[HTML]{ffC300} \\ 
    \hline
    Rare & \cellcolor[HTML]{74ff00} & \cellcolor[HTML]{74ff00} & \cellcolor[HTML]{fff000} & \cellcolor[HTML]{fff000} & \cellcolor[HTML]{ffC300} \\
    \hline
    \end{tabular}
    \caption{Risk Matrix (5x5)}
    \label{tab:risk_matrix}
\end{table}

\begin{comment}
    Mulige Risks

    Gruppemedlem blir syk, se tabell
    Miste source code eller dokumenter, som gjør at vi må recovere fra backups
    Infrastruktur og systemer blir utsatt for cyberangrep, typ ddos
    
\end{comment}

\begin{table}[H]
    \centering
    \begin{tabular}{|c|c|c|c|}
    \hline
    Risk \# & Description & Likelihood & Consequence \\
    \hline
    1 & \makecell{One or more group members get sick over a longer \\period, making it hard to work on the project.} & Unlikely \cellcolor[HTML]{ffC300} & Major \cellcolor[HTML]{ffC300} \\
    \hline
    2 & \makecell{Source code and/or documents become lost and unrecoverable \\due to  issues with their storage location.} & Unlikely \cellcolor[HTML]{ffC300} & Critical \cellcolor[HTML]{ffC300} \\
    \hline
    3 & \makecell{Infrastructure and systems could be attacked in a cyber attack, \\compromising the availability of the systems. } & Unlikely \cellcolor[HTML]{ffC300} & Major\cellcolor[HTML]{ffC300} \\
    \hline
    4 & \makecell{The project's objectives expand beyond the initial plan, \\ consuming additional time and resources} & Possible \cellcolor[HTML]{ffC300} & Major \cellcolor[HTML]{ffC300} \\
    \hline
    5 & \makecell{The sources for maps and geotechnical \\ data become inaccessible.} & Rare \cellcolor[HTML]{fff000} & Major \cellcolor[HTML]{fff000} \\
    \hline
    6 & \makecell{Delayed meetings or feedback from supervisor, \\ causing slight adjustments to the schedule.} & Possible \cellcolor[HTML]{74ff00} & Minimal \cellcolor[HTML]{74ff00} \\
    \hline
    \end{tabular}
    \caption{Risks}
    \label{tab:risks}
\end{table}

\begin{table}[H]
    \centering
    \begin{tabular}{|c|c|c|}
    \hline
    Risk \# & Mitigation \& Measures & Priority \\
    \hline
    1 & \makecell{The sick group member should work as much as possible. \\ Good documentation and communication will make it easier for \\ the other member to step in and help with the workload.} & Medium \cellcolor[HTML]{fff000} \\
    \hline
    2 & \makecell{Backups of all work should be made and stored in at least one \\separate location. See \ref{sec:documentation_storage_source_code}.} & High \cellcolor[HTML]{ff4233} \\
    \hline
    3 & \makecell{Proper and strict security rules needs to be in place to limit the \\access to only needed ports. Load-balancing and rate-limiting \\should be used to ensure availability.} & High \cellcolor[HTML]{ff4233} \\
    \hline
    4 & \makecell{Clearly define project goals and milestones. \\ Review the scope with stakeholders to prevent unnecessary additions.} & High \cellcolor[HTML]{ff4233} \\
    \hline
    5 & \makecell{Research alternate data sources early in the project and \\ implement modular design in the code for easier transition.} & High \cellcolor[HTML]{ff4233} \\
    \hline
    6 & \makecell{Practice good communication with the supervisor and \\ include buffer time in the project plan.} & Low \cellcolor[HTML]{74ff00} \\
    \hline
    \end{tabular}
    \caption{Risk Mitigation \& Measures}
    \label{tab:risk_mitigation}
\end{table}