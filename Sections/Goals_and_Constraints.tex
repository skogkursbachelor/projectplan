\section{Goals \& Constraints}
\subsection{Background}
Skogkurs is a non-governmental organization established in 1958. The institute operates as a partnership, with 36 forestry organizations and scientific institutions as its members. Its activities are nationwide and encompass a wide range of topics, including forest management, construction and maintenance of forest roads, and forest operations and techniques \cite{skogkurs_eng}. Through its initiatives, Skogkurs aims to enhance the competence of professionals within the forestry industry and to promote knowledge of forests and nature to schools and the general public \cite{skogkurs_nor}. 

The Nordic forestry industry faces increasing challenges in ensuring stable timber transport throughout the year. Changes in climate have extended the snow-free season, creating variability in forest road conditions due to shifts between frozen, dry, and rainy periods. Addressing these challenges requires digital tools that can assess which forest roads are accessible during different weather conditions.

Recent research has focused on developing methodologies for digitally classifying forest road load-bearing capacities based on soil types and weather conditions. For example, a nationwide pilot study demonstrated that soil types, such as well-drained materials like glacial deposits, are more suitable for year-round use, while finer sediments require specific conditions like freezing or drying. These findings support the potential for fully digital solutions to predict road accessibility based on a combination of weather patterns and road construction characteristics [Appendix \ref{appendix:thesis_proposal}].

\begin{comment}
    Additionally, tools like HarvesterSeasons.com, developed by the Finnish Meteorological Institute, provide weekly forecasts of road conditions based on soil moisture, temperature, and snow depth. These forecasts use data from sources like NASA's SMAP and ESA's Sentinel-1 satellites to generate relative load-bearing predictions for winter and snow-free seasons. Such advancements aim to increase efficiency and reduce risks in forest operations under variable weather conditions [\ref{appendix:thesis_proposal}.
\end{comment}

\subsection{Project Goals}
This section outlines the project's goals, which include product, impact, and learning goals.
\subsubsection{Product Goals}
% produkt som gjør ...
The primary goal of the project is to develop and test a prototype system for fully digital modeling of forest road load-bearing capacity under varying conditions throughout the year. The solution will incorporate various geological and meteorological parameters, such as soil type and moisture, weather forecasts, and precipitation data, to generate an accurate classification of forest roads. This classification will be presented through an interactive map-based website for easy accessibility. The roads will be color-coded using a traffic-light system, where green indicates safe roads, yellow signals caution, and red highlights unsafe roads. The system will provide users with a forecast for road conditions at least a week into the future, enabling better planning and decision-making. Furthermore, the system will prioritize ease of use, with an intuitive interface designed for transport managers, allowing them to make informed route choices based on real-time data and forecasts. 

\subsubsection{Impact Goals}
\begin{itemize}
    \item Reduced uncertainty for transport managers when setting the routes using forest roads.
    % kanskje increased effectiveness av transport
    \item To validate the prototype's feasibility and effectiveness by conducting tests with end-users, such as transport managers, to assess its performance and usability in real-world scenarios.
\end{itemize}

\subsubsection{Learning Goals}
% Map APIs / Working with Map Data
% More advanced APIs (?)
% Caching for REST APIs (?)
% Getting experience working with real companies, real products, etc.
% Terraform OpenStack
% More experience with Docker
% SDLC experience (SCRUM / agile methodology)
% Using professional workflow tools (?)
% Testing for specific features (?)
% Foresting and/or other fields (?)
% Planning full project self (finding req-spec., etc.)
% User testing (?)
% Concurrency

\begin{itemize}
    \item Gaining insight in implementing interactive maps and geospatial data on web pages.
    \item Leveraging RESTful APIs for efficient data integration.
    \item Acquiring hands-on experience collaborating with real-world companies and products.
    \item Developing a deeper understanding of the software development lifecycle while actively practicing agile methodologies, like Scrum.
    \item Enhancing application performance by implementing concurrency and optimizing parallel processing.
    \item Expanding proficiency in containerization techniques, particularly through hands-on experience with Docker.
    \item Implementing OpenStack deployment and configuration using Terraform for efficient infrastructure management.
    \item \textit{\textbf{From NTNU \cite{ntnu_idatg2900}:}}
    \begin{itemize}
        \item Has in-depth knowledge of a selected topic within the subject area.
        \item Has knowledge of research and development work within the topic.
        \item Can identify, formulate and solve a relevant engineering problem.
        \item Can apply knowledge and relevant results from research and development work to solve theoretical, technical and practical problems within the topic of the bachelor thesis and justify their choices.
        \item Can apply engineering methods and work methodically.
        \item Can document and disseminate engineering work.
        \item Can plan and carry out engineering work.
        \item Disseminates professional knowledge to various target groups both in writing and orally in Norwegian and English.
        \item Has insight into scientific honesty and understanding of ethical issues.
        \item Has insight into environmental, health, social and economic consequences of products and solutions within their field and can put these in an ethical perspective and a life cycle perspective.
        \item Integrates previously acquired knowledge and is able to acquire new knowledge in solving a problem.
    \end{itemize}
\end{itemize}

\subsection{Constraints}
\subsubsection{Temporal Constraints}
\begin{itemize}
    \item The set deadline for the final report is 20th of May.
    \item The presentation of the bachelor's thesis is scheduled for 4th or 5th of June.
\end{itemize}

\subsubsection{Product Constraints}
\begin{itemize}
    \item The product will be using HTML 5, which requires newer versions of browsers.
    \item The product must be able to integrate with existing APIs.
    \item The product must be accessible to users with varying levels of technical expertise, such as transport managers, requiring an intuitive and user-friendly interface.
\end{itemize}

\subsubsection{Legal Constraints}
\begin{itemize}
    \item The product must comply with the licensing terms and conditions of all third-party services, including map distributors, external APIs, and any code libraries, frameworks, or tools used in its development and deployment.
\end{itemize}
